% Author - Jon Arnt Kårstad, NTNU IMT
\documentclass{article}

% Importing document settings from our file "packages.sty"
\usepackage{packages}

% Beginning of document
\begin{document}

% Inserting title page
\import{./}{title}

% Defining front matter settings (Norsk: innstillinger for forord m.m.)
\frontmatter

% Inserting table of contents
\tableofcontents

% Inserting list of figures & list of tables
\listoffigures
\listoftables

% Defining main matter settings (Norsk: innstillinger for hoveddelen av teksten)
\mainmatter

% Introduction explaining this LaTeX-template
\section{Introduksjon og prosjektbeskrivelse}

Denne rapporten tar for seg design og implementering av en Universal Asynchronous Receiver Transmitter (UART). Modulen skal oppfylle gitte prosjektkrav, og skal implementeres i VHDL kode for DE10-Lite, og testes på dette FPGA kortet samt opp mot en annen mikrokontroller for å verifisere at UART-modulen har ønsket funksjonalitet og kan leses av andre enheter. Rapporten skal dokumenteres som et datablad. 

\section{Spesifikasjoner}

I dette prosjektet er det gitt spesifikasjoner for oppførselen til UART-modulen. Den skal bestå av en RX- og TX-modul, samt en kontrollmodul, CTRL, som vil være en styringsenhet for RX- og TX-modulene. Det er gitte spesifikke krav for hver modul, men overordnet for alle modulene er det krav om et 50MHz klokkesignal og et reset-signal som initialiserer modulene. Følgende tabeller er gitt for de ulike modulene. 

\begin{figure}[H]
    \centering
    \includegraphics[width=0.75\linewidth]{Images/rxkrav.png}
    %\caption{Enter Caption}
    \label{fig:rxkrav}
\end{figure}

\begin{figure}[H]
    \centering
    \includegraphics[width=0.75\linewidth]{Images/txkrav.png}
    %\caption{Enter Caption}
    \label{fig:txkrav}
\end{figure}

\begin{figure}[H]
    \centering
    \includegraphics[width=0.75\linewidth]{Images/ctrlkrav.png}
    \caption{Enter Caption}
    \label{fig:placeholder}
\end{figure}

\section{Teori}






\section{Dokumentasjon}


\subsection{Introduction}
%• En introduksjon av det implementerte designet med en oppsummering av funksjonalitet ogvirkem˚ate.

\subsubsection{Block diagram}

\subsubsection{}

%%• Tilstandsdiagrammer for eventuelle tilstandsmaskiner i implementasjonen og forklaring av
%disse.

\subsection{Receiver Module (RX)}


\subsection{Transmitter Module (TX)}


\subsection{Control Module (CTRL)}


%• Tidsdiagrammer og simuleringsresultater med forklaring/diskusjon.
%• Resultater fra syntetisering.
%• Dokumentasjon av testoppsett og resultater fra testing med mikrokontroller.
%• En oversikt over hvilke av de definerte kravspesifikasjonene som er oppfylt med kommen-
%tarer.
%• Det bør ogs˚a foreligge en forklaring om hva slags normer som har blitt fulgt under skriving
%av VHDL-koden.
%Det trengs ikke noen inng˚aende forklaring av UART-protokollen i dokumentet. Videre skal
%all kode, b˚ade for UART-modulen, testbenker og mikrokontroller legges ved i et eget arkiv. Det
%skal ogs˚a legges ved en video viser virkem˚aten til den implementerte modulen i samhandling med
%en mikrokontroller.

% Example section added directly into the main-file
\section{Conclusion}
.

\begin{figure}[H]
    \centering
    \includegraphics[width=0.75\linewidth]{Images/vurdering.png}
    \caption{Enter Caption}
    \label{fig:placeholder}
\end{figure}

% Printing bibliography
\newpage
\printbibliography[heading = bibintoc, title = Bibliography]    % 'bibintoc' inserts our bibliography into the table of contents

% Inserting appendix with separate settings
%\addappendix
%\import{./Appendices/}{example_appendix}

% End of document
\end{document}
